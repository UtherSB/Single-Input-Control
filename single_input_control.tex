\documentclass[12pt]{article}
\usepackage{amsmath}
\usepackage{amsthm}
\usepackage{amssymb}
\usepackage{graphicx}
\usepackage{multicol}
\usepackage[dvips,letterpaper]{geometry} % Margins.
\usepackage{hyperref}
\usepackage{comment}

%\makeatletter % Need for anything that contains an @ command 
%\renewcommand{\maketitle} % Redefine maketitle to conserve space
%{ \begingroup \vskip 10pt \begin{center} \LARGE {\emph \@title}
%	\vskip 40pt \large \@author \vskip 10pt \@date \end{center}
% \vskip 10pt \endgroup \setcounter{footnote}{0} }
%\makeatother % End of region containing @ commands

\let\underdot=\d % rename builtin command \d{} to \underdot{}
\renewcommand{\d}[2]{\frac{d #1}{d #2}} % for derivatives
\newcommand{\dd}[2]{\frac{d^2 #1}{d #2^2}} % for double derivatives
\newcommand{\pd}[2]{\frac{\partial #1}{\partial #2}} % for partial derivatives
\newcommand{\pdd}[2]{\frac{\partial^2 #1}{\partial #2^2}} % for double partial derivatives
\newcommand{\pdc}[3]{\left( \frac{\partial #1}{\partial #2}\right)_{#3}} % for thermodynamic partial derivatives
\newcommand{\avg}[1]{\langle #1 \rangle}
\newcommand{\ket}[1]{| #1 \rangle}
\newcommand{\bra}[1]{\langle #1 |}
\newcommand{\braket}[2]{\langle #1 | #2 \rangle} % for Dirac brackets
\newcommand{\ketbra}[2]{| #1 \rangle\langle #2 |}
\newcommand{\matrixel}[3]{\langle #1 | #2 | #3 \rangle} % for Dirac matrix elements

\def\dbar{{\mathchar'26\mkern-12mu d}}

\newtheorem{theorem}{Theorem}
\newtheorem{proposition}[theorem]{Proposition}
\newtheorem{lemma}[theorem]{Lemma}
\theoremstyle{plain}
\newtheorem{definition}[theorem]{Definition}
\theoremstyle{remark}
\newtheorem*{remark}{Remark}
\theoremstyle{plain}
\newtheorem{corollary}[theorem]{Corollary}
\newtheorem*{example}{Example}

\DeclareMathOperator{\SL}{SL}
\DeclareMathOperator{\SO}{SO}
\DeclareMathOperator{\SU}{SU}
\DeclareMathOperator{\Spin}{Spin}
\DeclareMathOperator{\Oh}{O}
\DeclareMathOperator{\eS}{S}

\usepackage{bbm}
\usepackage{enumerate}
\newcommand{\behaviour}{(\Xi,\widetilde{\mathcal{M}}, P)}
\DeclareMathOperator{\proj}{proj}
\DeclareMathOperator{\st}{s.t.}
\DeclareMathOperator{\Span}{span}
\DeclareMathOperator{\Tr}{Tr}

\begin{document}
\title{Single input, 2 mode counterexample to Alessio's conjecture using Jurdjevic and Kupka}
\author{Uther Shackeley-Bennett}
\maketitle

\section{Jurdjevic amd Kupka's Theorem}
From Elliott p95 and p117.
\begin{definition}[Strong regularity] The matrix $B \in \mathfrak{gl}(n, \mathbb{R})$ is strongly regular if its eigenvalues $\lambda_k = \alpha_k +i\beta_k$, $k \in 1, \ldots, n$ are distinct, including $2m$ conjugate-complex
pairs, and the real parts $\alpha_1 < \ldots < \alpha_{n-m}$ satisfy $\alpha_i - \alpha_j \neq \alpha_p - \alpha_q$ unless $i = p$ and $j = q$.
\end{definition}

\begin{theorem}[Jurdjevic and Kupka (note the capitalisation typo in Elliott)]
 Assume that $\operatorname{Tr}(A) = 0 = \operatorname{Tr}(B)$ and $B$ is strongly regular. Choose coordinates so that $B = \operatorname{diag}(\lambda_1, \ldots, \lambda_n)$. If $A$ satisfies 
\begin{align}
 A_{i,j} \neq 0 \quad \forall i,j &\text{ such that } |i - j| = 1, \\
A_{1,n}A&_{n,1} < 0
\end{align}
then with $\Omega = \mathbb{R}$, $\dot{x} = (A + uB)x$ is controllable on $\mathbb{R}^n_*$.
\end{theorem}

\begin{proposition}[From Meyer]
 The characteristic polynomial of a real Hamiltonian matrix is an even polynomial. Thus, if $\lambda$ is an eigenvalues of a Hamiltonian matrix, then so are $-\lambda, \lambda^*, -\lambda^*$. The trace of a Hamiltonian matrix id always $0$.
\end{proposition}

Note that any matrix with dinstinct eigenvalues is diagonalisable.

\textbf{I am wondering about why we are told to diagonalise B; I don't think the alphas in the theorem are the same as those in the definition.. So it just seems like a poor choice of notation in Elliott. It's annoying because it looks like we could be requiring that $B$ have real eigenvalues. I have changed my notation but I want this to be at the back of my mind.}

\section{Two mode, Hamiltonians matrix with the Jurdjevic-Kupka conditions}
In two modes a general Hamiltonian matrix is of the form
\begin{equation}
A = \begin{pmatrix} a & b & c & d \\ e & f & d & h \\k & l & -a & -e \\l & g & -b & -f \end{pmatrix}.
\end{equation}
Setting this as the form of our drift field we may then add the Jurdjevic conditions: $b,d,e,l \neq 0$ and $dl < 0$.

With its distinct eigenvalues $B$ can be diagonalised to one of three forms:
\begin{equation}
B_1 = \begin{pmatrix} x+iy & 0 & 0 & 0 \\ 0 & x-iy & 0 & 0 \\0 & 0 & -x+iy & 0 \\0 & 0 & 0 & -x-iy \end{pmatrix},
\end{equation}
\begin{equation}
B_2 = \begin{pmatrix} x & 0 & 0 & 0 \\ 0 & -x & 0 & 0 \\0 & 0 & y & 0 \\0 & 0 & 0 & -y \end{pmatrix},
\end{equation}
\begin{equation}
B_3 = \begin{pmatrix} ix & 0 & 0 & 0 \\ 0 & -ix & 0 & 0 \\0 & 0 & iy & 0 \\0 & 0 & 0 & -iy \end{pmatrix},
\end{equation}
for $x,y \in \mathbb{R}$.
The conditions that it must satisfy to be strongly regular are trivially satisfied in two modes.

Therefore a general $A+uB$ matrix looks like one of
\begin{equation}
G_1 = \begin{pmatrix} a+u(x+iy) & b & c & d \\ e & f+u(x-iy) & d & h \\k & l & -a+u(-x+iy) & -e \\l & g & -b & -f+u(-x-iy) \end{pmatrix},
\end{equation}
\begin{equation}
G_2 = \begin{pmatrix} a+ux & b & c & d \\ e & f-ux & d & h \\k & l & -a+uy & -e \\l & g & -b & -f-uy \end{pmatrix},
\end{equation}
\begin{equation}
G_3 = \begin{pmatrix} a+iux & b & c & d \\ e & f-iux & d & h \\k & l & -a+iuy & -e \\l & g & -b & -f-iuy \end{pmatrix}.
\end{equation}

\section{Neutrality}
\begin{equation}
 \dot{X} = AX + \sum_{1}^m u_i(t)B_iX, \quad X(0)=I_n, \quad u(\cdot)=\Omega, \quad t \geq 0. \label{eq:opcont}
\end{equation}



\begin{definition}[Elliott p97]
Neutrality of matrix $C$ is equivalent to each of the following properties:
\begin{itemize}
 \item $C$ is similar over $\mathbb{R}$ to a skew-symmetric matrix.
\item $\operatorname{spec}(C)$ lies in the imaginary axis and $C$ is diagonalisable over $C$.
\item The closure of $e^{\mathbb{R}C}$ is compact.
\item There exists $\sigma > 0$ and a sequence of times $\{t_k\}_1^\infty$ with $t_k \geq \sigma$ such that $\lim_{k\to\infty}||e^{t_kC}-I|| = 0$.
\item $| \Tr(C)| = 0$ and $||e^{tC}||$ are bounded on $\mathbb{R}$.
\end{itemize}
\end{definition}

\begin{theorem}
Let $\Omega = \mathbb{R}^m$
\begin{itemize}
 \item If there are constants $\delta_i \in \Omega$ such that $A_\delta := A+\sum_1^m\delta_iB_i$ is neutral then $S = G_\alpha$;
\item If, also, the Lie rank of $\mathfrak{g}_\alpha$ is $n$ then (\ref{eq:opcont}) is controllable.
\end{itemize}
\end{theorem}
\begin{proof}
 see Elliott p97.
\end{proof}

\subsection{Neutrality and Hamiltonian matrices}



% \section{Counterexample}
% Counterexample to neutrality being necessary and sufficient for symplectic algebra generators.
% 
% \subsection{Commutator approach}
% 
% The condition of neutrality for a Hamiltonian matrix $X$ is equivalent to $[J,X] = 0$ where $J$ is.... Consdiering $G_1$:
% \begin{equation}
% [J,G]=
% \begin{pmatrix} c+k & d+l & -2(a+ux) & -b-e \\ d+l & g+h & -b-e & -2(f+ux) \\ -2(a+ux) & -b-e & -c-k & -d-l \\ -b-e & -2(f+ux) & -d-l & -g-h \end{pmatrix}
% \end{equation}
% 
% For a counterexample we require that $b,d,e,l \neq 0$, $dl <0$ and \textbf{probably} that $x, y \neq 0$ too \textit{as well as} $[J,G] \neq 0 \; \forall u$. $u$ is only affecting certain of the elements so the initial answer seems to be yes, certainly.
% 
% \subsection{Old eigenvalue approach}
% I would like the eigenvalue equations of $A+uB$. From this my question is: are there values of $a,b,c,d,e,f,g,h,k,l,x,y$ such that there is no value of $u$ to make the eigenvalue expression less than $0$. I can have this function in matlab and then have a load of inputs so I can play around. Ensuring that I stick to the few conditions: that some can't be zero and the $dl$ condition.
% 
% \section{Examples}
% \subsection{Example 1}
% \begin{equation}
%  A = \begin{pmatrix} 0 & a & 0 & x \\ b & 0 & c & 0 \\0 & d & 0 & e \\y & 0 & f & 0 \end{pmatrix} = \begin{pmatrix} 0 & a & 0 & c \\ b & 0 & c & 0 \\0 & d & 0 & -b \\d & 0 & -a & 0 \end{pmatrix} = \begin{pmatrix} 0 & 2 & 0 & -3 \\ 4 & 0 & -3 & 0 \\0 & 5 & 0 & -4 \\5 & 0 & -2 & 0 \end{pmatrix} 
% \end{equation}
% $xy < 0$. The letters are non zero.
% 
% The above is recursive.
% 
% \begin{equation}
%  A = \begin{pmatrix} 0 & a & 0 & c \\ b & 0 & c & 0 \\0 & d & 0 & -b \\d & 0 & -a & 0 \end{pmatrix} = \begin{pmatrix} 0 & 7 & 0 & -3 \\ 7 & 0 & -3 & 0 \\0 & 5 & 0 & -7 \\5 & 0 & -7 & 0 \end{pmatrix} 
% \end{equation}
% 
% This example has all real eigenvalues. If we construct B such that this cannot be made imag. then we are OK....
% 
% \begin{equation}
%  B = \begin{pmatrix} a+ib & 0 & 0 & 0 \\ 0 & a-ib & 0 & 0 \\0 & 0 & c+id & 0 \\0 & 0 & 0 & c-id \end{pmatrix} = \begin{pmatrix} 2+i3 & 0 & 0 & 0 \\ 0 & 2-i3 & 0 & 0 \\0 & 0 & -2+i3 & 0 \\0 & 0 & 0 & -2-i3 \end{pmatrix}
% \end{equation}
% 
% The eigenvalues are all distinct and $a-c$ are not equal to any other real parts subtracted. 
% 
% What I want is an example such that a compact drift field is not constructible. Presumably no point trying in 1 mode because of Wu's theorem.
% 
% \begin{equation}
%  A+uB = \begin{pmatrix} u(2+i3) & 7 & 0 & -3 \\ 7 & u(2-i3) & -3 & 0 \\0 & 5 & u(-2+i3) & -7 \\5 & 0 & -7 & u(-2-i3) \end{pmatrix} 
% \end{equation}
% This example is not recursive for $u=1$. I need to show that this is never recursive for any value of $u$. On MATLAB I find that there are values of u for which this has imaginary eigenvalues.
% 
% So trying for a more general example
% \begin{equation}
%  A = \begin{pmatrix} 0 & a & 0 & c \\ b & 0 & c & 0 \\0 & d & 0 & -b \\d & 0 & -a & 0 \end{pmatrix} = \begin{pmatrix} 0 & 7 & 0 & -3 \\ 7 & 0 & -3 & 0 \\0 & 5 & 0 & -7 \\5 & 0 & -7 & 0 \end{pmatrix} 
% \end{equation}

\end{document}