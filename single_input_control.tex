\documentclass[12pt]{article}
\usepackage{amsmath}
\usepackage{amsthm}
\usepackage{amssymb}
\usepackage{graphicx}
\usepackage{multicol}
\usepackage[dvips,letterpaper]{geometry} % Margins.
\usepackage{hyperref}
\usepackage{comment}

%\makeatletter % Need for anything that contains an @ command 
%\renewcommand{\maketitle} % Redefine maketitle to conserve space
%{ \begingroup \vskip 10pt \begin{center} \LARGE {\emph \@title}
%	\vskip 40pt \large \@author \vskip 10pt \@date \end{center}
% \vskip 10pt \endgroup \setcounter{footnote}{0} }
%\makeatother % End of region containing @ commands

\let\underdot=\d % rename builtin command \d{} to \underdot{}
\renewcommand{\d}[2]{\frac{d #1}{d #2}} % for derivatives
\newcommand{\dd}[2]{\frac{d^2 #1}{d #2^2}} % for double derivatives
\newcommand{\pd}[2]{\frac{\partial #1}{\partial #2}} % for partial derivatives
\newcommand{\pdd}[2]{\frac{\partial^2 #1}{\partial #2^2}} % for double partial derivatives
\newcommand{\pdc}[3]{\left( \frac{\partial #1}{\partial #2}\right)_{#3}} % for thermodynamic partial derivatives
\newcommand{\avg}[1]{\langle #1 \rangle}
\newcommand{\ket}[1]{| #1 \rangle}
\newcommand{\bra}[1]{\langle #1 |}
\newcommand{\braket}[2]{\langle #1 | #2 \rangle} % for Dirac brackets
\newcommand{\ketbra}[2]{| #1 \rangle\langle #2 |}
\newcommand{\matrixel}[3]{\langle #1 | #2 | #3 \rangle} % for Dirac matrix elements

\def\dbar{{\mathchar'26\mkern-12mu d}}

\newtheorem{theorem}{Theorem}
\newtheorem{proposition}[theorem]{Proposition}
\newtheorem{lemma}[theorem]{Lemma}
\theoremstyle{plain}
\newtheorem{definition}[theorem]{Definition}
\theoremstyle{remark}
\newtheorem*{remark}{Remark}
\theoremstyle{plain}
\newtheorem{corollary}[theorem]{Corollary}
\newtheorem*{example}{Example}

\DeclareMathOperator{\SL}{SL}
\DeclareMathOperator{\SO}{SO}
\DeclareMathOperator{\SU}{SU}
\DeclareMathOperator{\Spin}{Spin}
\DeclareMathOperator{\Oh}{O}
\DeclareMathOperator{\eS}{S}

\usepackage{bbm}
\usepackage{enumerate}
\newcommand{\behaviour}{(\Xi,\widetilde{\mathcal{M}}, P)}
\DeclareMathOperator{\proj}{proj}
\DeclareMathOperator{\st}{s.t.}
\DeclareMathOperator{\Span}{span}
\DeclareMathOperator{\Tr}{Tr}

\begin{document}
\title{Single input control and Jurdjevic's theorem}
\author{Uther Shackeley-Bennett}
\maketitle

\section{Jurdjevic amd Kupka's Theorem}
From Elliott p95 and p117.
\begin{definition}[Strong regularity] The matrix $B \in \mathfrak{gl}(n, \mathbb{R})$ is strongly regular if its eigenvalues $\lambda_k = \alpha_k +i\beta_k$, $k \in 1, \ldots, n$ are distinct, including $2m$ conjugate-complex
pairs, and the real parts $\alpha_1 < \ldots < \alpha_{n-m}$ satisfy $\alpha_i - \alpha_j \neq \alpha_p - \alpha_q$ unless $i = p$ and $j = q$.
\end{definition}

\begin{theorem}[Jurdjevic and Kupka]
 Assume that $\operatorname{Tr}(A) = 0 = \operatorname{Tr}(B)$ and $B$ is strongly regular. Choose coordinates so that $B = \operatorname{diag}(\alpha_1, \ldots \alpha_n)$. If $A$ satisfies 
\begin{align}
 A_{i,j} \neq 0 \quad \forall i,j &\text{ such that } |i - j| = 1, \\
A_{1,n}A&_{n,1} < 0
\end{align}
then with $\Omega = \mathbb{R}$, $\dot{x} = (A + uB)x$ is controllable on $\mathbb{R}^n_*$.
\end{theorem}

Note that our A and B are always traceless. I am wondering about why we are told to diagonalise B; I don't think the alphas in the theorem are the same as those in the definition.


\section{Examples}
\subsection{Example 1}
\begin{equation}
 A = \begin{pmatrix} 0 & a & 0 & x \\ b & 0 & c & 0 \\0 & d & 0 & e \\y & 0 & f & 0 \end{pmatrix} = \begin{pmatrix} 0 & a & 0 & c \\ b & 0 & c & 0 \\0 & d & 0 & -b \\d & 0 & -a & 0 \end{pmatrix} = \begin{pmatrix} 0 & 2 & 0 & -3 \\ 4 & 0 & -3 & 0 \\0 & 5 & 0 & -4 \\5 & 0 & -2 & 0 \end{pmatrix} 
\end{equation}
$xy < 0$. The letters are non zero.

The above is recursive.

\begin{equation}
 A = \begin{pmatrix} 0 & a & 0 & c \\ b & 0 & c & 0 \\0 & d & 0 & -b \\d & 0 & -a & 0 \end{pmatrix} = \begin{pmatrix} 0 & 7 & 0 & -3 \\ 7 & 0 & -3 & 0 \\0 & 5 & 0 & -7 \\5 & 0 & -7 & 0 \end{pmatrix} 
\end{equation}

This example has all real eigenvalues. If we construct B such that this cannot be made imag. then we are OK....

\begin{equation}
 B = \begin{pmatrix} a+ib & 0 & 0 & 0 \\ 0 & a-ib & 0 & 0 \\0 & 0 & c+id & 0 \\0 & 0 & 0 & c-id \end{pmatrix} = \begin{pmatrix} 2+i3 & 0 & 0 & 0 \\ 0 & 2-i3 & 0 & 0 \\0 & 0 & -2+i3 & 0 \\0 & 0 & 0 & -2-i3 \end{pmatrix}
\end{equation}

The eigenvalues are all distinct and $a-c$ are not equal to any other real parts subtracted. 

What I want is an example such that a compact drift field is not constructible. Presumably no point trying in 1 mode because of Wu's theorem.

\begin{equation}
 A+uB = \begin{pmatrix} u(2+i3) & 7 & 0 & -3 \\ 7 & u(2-i3) & -3 & 0 \\0 & 5 & u(-2+i3) & -7 \\5 & 0 & -7 & u(-2-i3) \end{pmatrix} 
\end{equation}
This example is not recursive for $u=1$. I need to show that this is never recursive for any value of $u$. On MATLAB I find that there are values of u for which this has imaginary eigenvalues.


So trying for a more general example
\begin{equation}
 A = \begin{pmatrix} 0 & a & 0 & c \\ b & 0 & c & 0 \\0 & d & 0 & -b \\d & 0 & -a & 0 \end{pmatrix} = \begin{pmatrix} 0 & 7 & 0 & -3 \\ 7 & 0 & -3 & 0 \\0 & 5 & 0 & -7 \\5 & 0 & -7 & 0 \end{pmatrix} 
\end{equation}


So what I want is to construct some Hamiltonian matrix $A+uB$ satisfying all the condtion on $A$ and $B$ such that there is no value of $u$ that will give it imaginary eigenvalues. If there is then we could just set this as our drift field. I want to construct a controllable counter-example to Alessio's conjecture.

\subsection{Master equation}
General $A$ matrix:
\begin{equation}
A = \begin{pmatrix} a & b & c & d \\ e & f & d & h \\k & l & -a & -e \\l & g & -b & -f \end{pmatrix}
\end{equation}
such that $b,d,e,l \neq 0$ and $dl < 0$. (note that without these conditions this is just a general Hamiltonian matrix).

General $B$ matrix:
\begin{equation}
 B = \begin{pmatrix} x+iy & 0 & 0 & 0 \\ 0 & x-iy & 0 & 0 \\0 & 0 & -x+iy & 0 \\0 & 0 & 0 & -x-iy \end{pmatrix}.
\end{equation}

General $A+uB$ matrix:
\begin{equation}
A = \begin{pmatrix} a+u(x+iy) & b & c & d \\ e & f+u(x-iy) & d & h \\k & l & -a+u(-x+iy) & -e \\l & g & -b & -f+u(-x-iy) \end{pmatrix}.
\end{equation}

I would like the eigenvalue equations of $A+uB$. From this my question is: are there values of $a,b,c,d,e,f,g,h,k,l,x,y$ such that there is no value of $u$ to make the eigenvalue expression less than $0$. I can have this function in matlab and then have a load of inputs so I can play around. Ensuring that I stick to the few conditions: that some can't be zero and the $dl$ condition.


\end{document}
